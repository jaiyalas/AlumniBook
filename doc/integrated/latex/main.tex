%!TEX TS-program = xelatex
\documentclass[11pt]{article}

\usepackage[a4paper,top=2cm, bottom=1.5cm, left=2cm, right=2cm]{geometry}

\usepackage{fontspec}
\usepackage{xeCJK} 

\XeTeXlinebreaklocale "zh"
\XeTeXlinebreakskip = 0pt plus 1pt

% \setCJKmainfont{LiHei Pro}
\setCJKmainfont{LiSong Pro}

\newfontfamily{\K}{Hiragino Maru Gothic Pro}

\begin{document}

\begin{center}{\Huge AlumniBook} {G3} \end{center}

\begin{center}
\begin{tabular}{cccc}
郗昀彥 R00725051&郭瀚智 R02725023&鄭立民 R02725041&李奕德 B99705021 \\
張凱涵 B00705027&施淮振 B00705047&倪嘉銘 T02705102&
\end{tabular}
\noindent\makebox[\linewidth]{\rule{\linewidth}{0.4pt}}
\end{center}

\section{專案背景及目的}

本系在發展茁壯的同時培養了許多傑出的人才,這些人散布在各個不同領域。在超級專業化的現代若能連結正確領域的人來提供系上一些建議與幫助,必能讓資管系所內的課程內容與訓練更加的紮實。所以除了學生間的相互聯絡,本系統也將會提供系友與教授及行政人員進入本系統,對其所擅長之專業領域提供協助。學生亦可藉由本系統盡早深入了解產業現況,獲得更多正確且即時的資訊並對未來發展方向有更確實精確的判斷。

為了確保使用者身分得正確性,本系統必須要求使用者經由計算機中心(或其他認證機構)所提供的身分認證服務來認證使用者和其所宣稱的人為同一個人以防冒用。經正確認證之使用者於進入本系統後,會來到共用之瀏覽區。在這個區塊可以看到目前系統中的所有議題,瀏覽並參與討論。在觀看議題的同時使用者也可針對特定議題發表評論,此評論將會由任何觀看此議題的人共同看見。

若在閱讀討論區時對特定人士產生興趣,也可以在適當的隱私權相關管理下查詢特定的系友的相關資訊。能否能夠看見該使用者所有相關資料取決於資料擁有者的個人隱私條件以及搜尋者之帳號類別,或是搜尋者與被搜尋者之間有在本系統中建立特別的朋友關係。本系統中若存在朋友關係則再搜尋時會由特定對象獨立於依照帳號類別做設計的隱私規則,直接套用該使用者針對朋友設定的隱私權相關設定,讓朋友間能夠更加深入的了解彼此。

除了瀏覽討論區外,使用者亦可編輯自身的相關資料與隱私相關選項來讓自己被更多人看見。本系統將提供預先填入之公開資訊、基礎的個人連絡資訊輸入、以及與職業生涯相關之記錄,三個主要區塊。使用者可依照自身之習慣與想法選擇性的填入資料與決定其公開性。預設的群組別為校友、學生、教授及行政人員、與院系人員及朋友幾種,未來可再依據使用者需求做調整。

本系統旨在於建立一個能夠有效凝聚資管系所內之相關人士,提供一個跨界跨部(研究所與大學部)之溝通平台。本系統將階段性提供完整的溝通整合服務,冀望成為連結具專業知識的人幫忙解答疑惑,也希望能夠成為校友找人、學生找事之合作橋梁。

\section{專案假設與限制}

\end{document}
