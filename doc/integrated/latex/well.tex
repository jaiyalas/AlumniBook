\section{課程心得}
\subsection{郗昀彥 R00725051}

\subsection{郭瀚智 R02725023}

\subsection{鄭立民 R02725041}

於本學期的課程專案中,首先在設計階段學到了許多的UML圖,雖然應用上也只有畫USE CASE跟Sequence而已,但其實許多其他圖的精神也都有包還在內(像是Sequence感覺跟活動流程圖還蠻像的),當然能各種圖都畫出來是最好,也最容易無死角的了解。相信在未來有做專案設計的機會時,這些經驗與知識會有所幫助。

同時在這門課也教了各式設計巧思的Design pattern。雖然沒實際運用過的話,說實在也很難記得。其他還有像是網站資安問題,其中各式各樣的SQL injection讓我最為印象深刻。

本次專案最重要的是,真的體悟到一份專案要管理好實在十分困難,各式各樣的時間不好喬,各式各樣的溝通,加上組員間的能力以及積極度差異,皆會增加做好專案的難易度。
所幸我們有強大的Coding大師們以及組長的積極溝通協調,上山下海全都包,否則產出一份專案當真沒那麼容易。

\subsection{李奕德 B99705021}

在這學期的課程中我學到的Design pattern和Datamodeling令我受益最多,這門課著實地讓我知道我從過去到現在所有寫過的東西都非常的雜亂且狹隘。Android系統的開發上,雖然課堂時間有限所以很緊湊的上了非常多的東西,不過仍然對我們後來再寫手機端時仍然很有幫助。在課堂中所學將我們會用到比較複雜的部份都有帶到,讓開發起來不至於完成不了任何要求。雖然由於時間壓力沒辦法將所有設計都實作,但是仍完成了比較核心的部分。

在這次的project中我們選擇了MVC切割較為清楚的ruby on rails來實作,在製作Android端程式時也將三種不同的面相分成不同的thread去處理以求落實MVC架構。同時,我們盡量在規劃資料模型時讓他具有一定的彈性來使未來需要擴充時不至於完全沒辦法做。

\subsection{張凱涵 B00705027}

\subsection{施淮振 B00705047}

\subsection{倪嘉銘 T02705102}

暑假在家裡選課的時候就對軟體開發這門課程充滿期待,因為在重慶大學的資管系並沒有開設類似的寫程序的課程,然而我對軟體開發還是比較有興趣的,所以很期待這門課程。

在一個學期的學習中,課程的進度顯然要比大陸的的課程快很多,例如我們第一個作業就要求用兩個禮拜完成一個簡單的demo,然而老師在整個過程中都沒有教任何的程式,完全讓學生自己研究。這一點,大陸很難做到,但是其實我還是覺得台灣這樣的教育是非常好的,一方面這樣的方式能夠充分鍛煉學生的自學能力,另一方面也能讓整個課程更加充實,老師能夠講授更多有意義的東西。

在整個學期的過程中,老師邀請的嘉賓從各個方面系統地全面地給我們介紹了軟體開發的各個方面。例如一開始老師就非常強調工具的使用,比如git,這也是在大陸的課程中沒有接觸到的一部分。我能感受的區別就是,台灣的資管系與業界的聯繫要比大陸的資管系課程與業界的練習緊密許多,但是無疑台大這樣的方式能夠讓學生盡早地接觸與業界相關的東西,更能夠適應業界的規則,也能夠讓學生更快地進步。許多嘉賓的講課讓我在軟體開發方面看到了更遠的東西,不再僅僅是將老師要求的功能實現,還有更深遠的程序質量,程序後期的維護,程序的變動,數據模型的合理建構等等。

在project方面,老師要求我們分組做一個校友管理的系統,便是要求我們實作一個程序,我們組所應用的現在漸漸流行的ROR,也讓我接觸到了,業界更新的,更酷的技術,雖然自己在整個開發過程中都出於學習的狀態,並沒有特別多的實作貢獻,但是還是學習到了很多心的東西,收益良多。

有一點比較遺憾的就是之前乜有學習,面向對象的C++和java,所以有些課程並沒有能夠很好地掌握,所以學習效果似乎並沒有那麼好。

總而言之,一個學期的學習還是十分充實的。

