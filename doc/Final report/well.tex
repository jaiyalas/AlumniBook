\section{課程心得}
\subsection*{郗昀彥 R00725051}

我在這門課程中學習到最重要的是資料模型的抽象化。過往我都是自己看書與摸索關於程式架構的抽象化,而鮮少意識到資料模型的重要性。在這次學期專案中我有限程度地嘗試了如何去抽象化資料模型,這過程中發現一些非常有趣和是也非常熟悉的問題。首先,資料模型的抽象化本質上和程式架構的抽象化有相當程度的相似性。這似乎也不是很意外,畢竟所謂的抽象化在某個觀點下可以被簡單地理解成是一種尋找最大共識的過程。我們期望我們的程式架構在可預期的需求上具有最大的相容可能性,同樣地,資料模型也是。此外,隨著抽象層級升高,各種常見的資料或程式的設計模式就被發現或是發明出來。此外,兩種抽象化之間也存在著些許共通的問題,例如說過度設計 (over-design)。

這次專題中,在我構思我們的資料模型時就可以時時刻刻感受到過度設計一直在某個陰暗的角落隨時等著撲上來。尤其是當我們開始用相當高階的角度去描述"關係"這種可以另外詮釋的議題時。例如說本次專案中我們使用的「關注」關係,「關注」(following) 理想上初步有兩種實作,一種是單向的「訂閱」 (subscribing) 另外一種是雙向的「朋友」 (firendship)。兩種關係都被抽象成是某種「關注」,而理想上可以把各種不同的「關注」的詮釋(或是定義)也從程式中抽象成規則庫 (rule database),如此一來整體可以享有的抽象化強度又更加完整。這在設計上會是個好選擇,但是實務上這會牽涉到很深入的程式撰寫議題,甚至是一些尚不完全成熟的自動化以及半自動化編程的技術。換句話說,這樣的過度抽象化就直接造成了系統過度複雜而會在另一個方向產生難題。

總結來說,不管是我自己熟悉的 GoF 設計模式或函數式設計模式以及資料模型的設計模式,這些抽象化技巧的取捨與平衡是我在這學期的課程中最大的收穫。而這些各式各樣的抽象化也或許是資訊相關學系最重要的議題之一,也希望未來會有機會研究這些抽象技術之間的關係。

\subsection*{郭瀚智 R02725023}

這次課程讓我深刻體會到理解使用者需求的重要性,隨著專案開發的過程,系統修改成本隨著時間提昇,敏捷式開發可以透過每一個回合快速檢視出系統不符合需求的地方並加以改進,這次透過三個回合從設計到實作,在短短幾個月中終於完成一個小雛形,在過程中反覆修正與分析功能的重要性與人力時間的分配,讓我想起『人月神話:軟體專案管理之道』這本軟體專案開發身具影響力的經典之作,軟體開發式需要大量溝通的過程,並不像種田一樣,加越多人進來進度就越快,相反的因為人越多溝通成本越大,擁有良好的溝通方式與互動模式有助於專案開發效率,反覆測試也是必要的流程,包含程式驗證與使用者測試,在軟體開發中沒有完美的解決之道,而是在時間、成本、品質三者之中取其平衡點並且戰且走,推向市場更是一條無盡又具挑戰性的旅途。

\subsection*{鄭立民 R02725041}

於本學期的課程專案中,首先在設計階段學到了許多的UML圖,雖然應用上也只有畫USE CASE跟Sequence而已,但其實許多其他圖的精神也都有包還在內(像是Sequence感覺跟活動流程圖還蠻像的),當然能各種圖都畫出來是最好,也最容易無死角的了解。相信在未來有做專案設計的機會時,這些經驗與知識會有所幫助。

同時在這門課也教了各式設計巧思的Design pattern。雖然沒實際運用過的話,說實在也很難記得。其他還有像是網站資安問題,其中各式各樣的SQL injection讓我最為印象深刻。

本次專案最重要的是,真的體悟到一份專案要管理好實在十分困難,各式各樣的時間不好喬,各式各樣的溝通,加上組員間的能力以及積極度差異,皆會增加做好專案的難易度。
所幸我們有強大的Coding大師們以及組長的積極溝通協調,上山下海全都包,否則產出一份專案當真沒那麼容易。

\subsection*{李奕德 B99705021}

在這學期的課程中我學到的Design pattern和Datamodeling令我受益最多,這門課著實地讓我知道我從過去到現在所有寫過的東西都非常的雜亂且狹隘。Android系統的開發上,雖然課堂時間有限所以很緊湊的上了非常多的東西,不過仍然對我們後來再寫手機端時仍然很有幫助。在課堂中所學將我們會用到比較複雜的部份都有帶到,讓開發起來不至於完成不了任何要求。雖然由於時間壓力沒辦法將所有設計都實作,但是仍完成了比較核心的部分。

在這次的project中我們選擇了MVC切割較為清楚的ruby on rails來實作,在製作Android端程式時也將三種不同的面相分成不同的thread去處理以求落實MVC架構。同時,我們盡量在規劃資料模型時讓他具有一定的彈性來使未來需要擴充時不至於完全沒辦法做。

\subsection*{張凱涵 B00705027}

這學期修軟開其實是抱著一種想要嘗試學習更多東西的心態。我覺得這學期的課讓我獲益良多,而且是任何一堂皆然。在修這門課之前,我其實幾乎不會寫網頁,所以大部分東西都是第一次接觸。但是很多內容因為我實作經驗不夠多,導致有點無法吸收。整體而言,我了解了一個團隊合作開發應該如何運作,也學到自己第二次作業寫得有多少漏洞和問題XD

\subsection*{施淮振 B00705047}

這門課我學到最多的是關於一個專案的完整開發流程,還有與他人共同合作一個專案。以前寫的程式規模沒有那麼大,一個人就能完成,這次的專題是個很難得的經驗,學習到了一些業界常用的專案運行模式,並運用在專題上面,也學會用很多好用的開發工具,在專題上也發揮很好的效果。撰寫程式方面學會了新的思考方式以及實用的設計模式,讓我重新檢視自己所寫的程式,能更有效率地設計出一個程式,透過更好的架構設計程式也變得更有彈性,能很輕鬆的融合各種可能。修完這門課後相信我在軟體開發這個領域有了更進一步的了解與體悟。

\subsection*{倪嘉銘 T02705102}

暑假在家裡選課的時候就對軟體開發這門課程充滿期待,因為在重慶大學的資管系並沒有開設類似的寫程序的課程,然而我對軟體開發還是比較有興趣的,所以很期待這門課程。

在一個學期的學習中,課程的進度顯然要比大陸的的課程快很多,例如我們第一個作業就要求用兩個禮拜完成一個簡單的demo,然而老師在整個過程中都沒有教任何的程式,完全讓學生自己研究。這一點,大陸很難做到,但是其實我還是覺得台灣這樣的教育是非常好的,一方面這樣的方式能夠充分鍛煉學生的自學能力,另一方面也能讓整個課程更加充實,老師能夠講授更多有意義的東西。

在整個學期的過程中,老師邀請的嘉賓從各個方面系統地全面地給我們介紹了軟體開發的各個方面。例如一開始老師就非常強調工具的使用,比如git,這也是在大陸的課程中沒有接觸到的一部分。我能感受的區別就是,台灣的資管系與業界的聯繫要比大陸的資管系課程與業界的練習緊密許多,但是無疑台大這樣的方式能夠讓學生盡早地接觸與業界相關的東西,更能夠適應業界的規則,也能夠讓學生更快地進步。許多嘉賓的講課讓我在軟體開發方面看到了更遠的東西,不再僅僅是將老師要求的功能實現,還有更深遠的程序質量,程序後期的維護,程序的變動,數據模型的合理建構等等。

在project方面,老師要求我們分組做一個校友管理的系統,便是要求我們實作一個程序,我們組所應用的現在漸漸流行的ROR,也讓我接觸到了,業界更新的,更酷的技術,雖然自己在整個開發過程中都出於學習的狀態,並沒有特別多的實作貢獻,但是還是學習到了很多心的東西,收益良多。

有一點比較遺憾的就是之前乜有學習,面向對象的C++和java,所以有些課程並沒有能夠很好地掌握,所以學習效果似乎並沒有那麼好。

總而言之,一個學期的學習還是十分充實的。

